\documentclass[t,11pt,british,english, top=1.0in]{beamer}
\usepackage{hyperref}
\usepackage{graphics}
\usepackage{amsmath}
\usepackage{amssymb}
\usepackage{natbib}
        %You can use the package \textbf{pgfpages} 
        %to arrange your slides for printing. This is also explained
        %in the \textbf{beamer} documentation.

\newcommand{\tensor}[1]{\overline{\overline{#1}}}
\newcommand{\tautens}{\tensor{\tau}}

\setbeamerfont{framesubtitle}{size=\normalsize}
\setbeamerfont{framesubtitle}{size=\normalsize}
%\setbeamertemplate{frametitle}[default][center]
%\setbeamersize{text margin left=6mm}

\usetheme{Madrid}
\usecolortheme{orchid}

%gets rid of bottom navigation bars
\setbeamertemplate{footline}[page number]{}
\setbeamertemplate{headline}{}

%gets rid of navigation symbols
\setbeamertemplate{navigation symbols}{}

\begin{document}
\title{Data Handling Workshop\\\vspace*{5mm}\small 4 March 2014}
\author{} 
\date{} 

\frame{\titlepage} 

\frame{
   \frametitle{What is data?}
   \framesubtitle{Definition} 
   \begin{itemize}
    \item Data is a {\color{red}set of values} corresponding to one or more {\color{red}quantitative} or {\color{red}qualitative variables}.
    \vspace*{6mm}
    \item Examples:
      \begin{itemize}
        \item Sea levels measured every hour at a fixed location\vspace*{1mm}
        \item Speed of a car throughout time\vspace*{1mm}
        \item Metadata (= data that describes other data) for webpages\vspace*{1mm}
        \item Wind velocity at different locations in the UK
      \end{itemize}
   \end{itemize}
}

\frame{
   \frametitle{Why is it important?}
   %\framesubtitle{Definition} 
   \begin{itemize}
    \item Data 
   \end{itemize}
}



\frame{
   \frametitle{Issues to consider}
   %\framesubtitle{Definition} 
   \begin{itemize}
    \item Data 
   \end{itemize}
}




\end{document}
