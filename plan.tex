\documentclass[a4paper,11pt]{article}
\usepackage[top=3cm, bottom=3cm, left=3cm, right=3cm]{geometry}
\usepackage[T1]{fontenc}
\usepackage[utf8]{inputenc}
\usepackage{lmodern}

\title{Data Handling Course for Researchers}
\date{4 March 2014}
\author{Grantham Institute for Climate Change, Imperial College London}

\begin{document}

\maketitle
\setlength{\parskip}{0.3cm}
\setlength{\parindent}{0cm}

\section{Course learning outcomes}
After the course, students will:
\begin{itemize}
   \item Understand what data is and why it is important.
   \item Be aware of best practices when handling and storing data.
   \item Understand the need to backup, compress, and encrypt data.
   \item Be aware of the tools available for analysis and testing of data.
   \item Be able to use a version-controlled file repository.
   \item Be able to analyse a data set.
\end{itemize}

\section{Topics covered}

\subsection{What is data?}
Data is a set of values corresponding to one or more quantitative or qualitative variables. Examples: .......

\subsection{Why is data important?}
Permits new scientific discoveries.

Data can come from existing sources, may be derived from several data sets, or generated.

Data provenance: where did the data originally come from? Can it be trusted?

Licensing: who can use the data, and how?

\subsection{Best Practices}
Know the Imperial College data backup policy.
Always keep several regular backups, far apart from each other (not in the same building).
Responsibility to encrypt sensitive information, both locally and when transferring data to another location.

\subsection{Data storage}

Cloud services 

\subsection{Tools}


\subsection{Exercise}
Show the students how to create an account on GitHub.

With a data set comprising tidal data, demonstrate performing a harmonic analysis in Python, produce a plot of the tidal constituents, and add the plot to GitHub.


\end{document}
